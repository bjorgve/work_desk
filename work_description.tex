\documentclass[12pt]{article}
\usepackage{parskip}
\usepackage[superscript, biblabel]{cite}
\usepackage[nolist]{acronym}

\begin{acronym}
  \acro{DFT}{Density Functional Theory}
  \acro{HF}{Hartree Fock}
  \acro{SCF}{self-consistent field}
  \acro{MW}{Multiwavelet}
  \acro{MRCPP}{MultiResolution Computation Program Package}
  \acro{MRChem}{MultiResolution Chemistry}
  \acro{vampyr}[VAMPyR]{Very Accurate Multiresolution Python Routines}
  \acro{KS}{Kohn-Sham}
  \acro{API}{Application Programming Interface}
\end{acronym}

\begin{document}

\title{Tutorial package for VAMPyR}

\maketitle

\section{Background}

A recent development in quantum chemistry is the use of real-space grid
methods to achieve high precision results within the framework of
\ac{DFT}. One such method makes use of \acp{MW} as a
basis\cite{harrison2003multiresolution}
and is currently being used and further developed at the Hylleraas
center in Troms\o \, in the group of Ass.~Prof.~Luca Frediani\cite{frediani2013fully}.

The group has developed two codes the last 10 years the \ac{MRChem}\cite{mrchem} and
the \ac{MRCPP}\cite{mrcpp}.
\ac{MRChem}, is
a numerical real-space code for molecular electronic structure calculations
within the \ac{SCF} approximation of quantum chemistry. This is built on
\ac{MRCPP}, which is a general purpose numerical mathematics library
based on multiresolution analysis and the \ac{MW} basis which provide
low-scaling algorithms as well as rigorous error control in numerical
computations. These codes are written in C++, a low level programming language.
C++ developed codes are both fast and robust, but for visualization and
rapid code development it is inferior Python.

To facilitate rapid code development,
the \ac{MRChem} group has started to develop a third code,
\ac{vampyr}\cite{vampyr}. \ac{vampyr} gives access to the
fast and parallel \ac{API}
of \ac{MRChem} and \ac{MRCPP} from within Python. Using Python developers can
combine \ac{vampyr} with libraries such as NumPy for linear algebra, matplotlib\cite{matplotlib}
and avogadro\cite{avogadro} for visualization of orbitals, densities, molecular structures
and unit-cells.

\section{Project goals}

We will design and implement a set of tutorials and examples
using Jupyter Notebooks\cite{jupyter} on how
\ac{vampyr} can be integrated with \ac{MRChem} and \ac{MRCPP} to
facilitate code development and to provide powerful analysis tools for
the end-user, such as for example, visualizing orbitals, densities,
molecular structures and unit-cells.

As a starting point we will consider the following types of tutorials.

\begin{enumerate}
    \item Tutorial of how orbitals, densities, molecular structure and unit cells
    can be visualized using matplotlib, openchemistry and/or avogadro.
    \item Tutorial of the mathematical tools in \ac{vampyr}.
    \item Tutorial of how the basic quantum chemical tools of \ac{vampyr}.
    \item Tutorial of showing how the mathematical tools can be combined
    with the quantum mechanical tools to solve the
    the \ac{HF} and \ac{KS} equations.
    \item Tutorial of the more advanced solvers can be used to develop new
    applications.
\end{enumerate}

The first one is aim at the end-user whereas the other two are more
for program developers.

\section{Implementation}

To begin with we need to familiarize ourselves with the visualization packages in
Python.
Then \ac{vampyr} needs to be extended to such that it generates output objects
that can be used by the visualization packages as inputs to display orbitals,
densities, molecular structures and unit-cells. When this is done the tutorial
will contain guides on how to install and import the packages to Python.
Then create examples of how the different
packages can be combined with \ac{vampyr} to visualize, such as, orbitals,
densities, molecular structures and unit-cells.

The tutorial on the mathematical tools in \ac{vampyr} should begin
with introducing the mathematical concepts the code, such as
the function representations, arithmetic operations and integral
operators. The tutorial should end in a theoretical discussion
on how to solve \ac{KS}-like equations self-consistently using
integral operators, and an example of how this can be done
by \ac{vampyr}.

For the tutorial on how \ac{vampyr} can be used to solve the restricted-\ac{HF},
the tutorial should begin with a theoretical overview covering the theoretical
aspects of the implementation. Then an example system should be implemented
to calculate the energy on a simple molecular system.

The work will be supervised by Ass.~Prof.~Luca Frediani and Research Fellow
Magnar Bj\o rgve.

\section{Budget}

The project will be carried out in the period May-July 2019. 
The budget for the project will cover travel expenses, accommodation
expenses and participation to the ``ISTCP X'' conference.

\begin{table}[!htb]
  \begin{tabular}{l|r|r|r}
               & Price (NOK) & Qt & Total (NOK)\\
\hline
    Travel        & 10.000      & 1  & 10.000     \\
    Accommodation &  8.000      & 3  & 24.000     \\
    ISTCP X       &  5.000      & 1  &  5.000     \\
\hline
    Total         &             &    & 39.000     \\
  \end{tabular}
  \caption{Budget plan}
  \label{tab:budget}
\end{table}

\bibliographystyle{unsrt}
\bibliography{ref.bib}
\end{document}
