\documentclass[12pt]{article}
\usepackage{parskip}
\usepackage[superscript, biblabel]{cite}
\usepackage{acronym}

\begin{acronym}
  \acro{DFT}{Density Functional Theory}
\end{acronym}

\begin{document}

\title{Bla bla}

\maketitle

\section{Background}

A recent development in quantum chemistry is the use real-space grid
methods to achieve high precision results within the framework of
\ac{DFT}. One such method makes use of Multiwavelet as a basis\cite{}
and is curently being used and further developed at the Hylleraas
centre in Tromsø in the group of Ass.~Prof.~Luca Frediani.

\section{Project goals}

\section{Implementation}

The Very Accurate Multiresolution Python Routines (VAMPyR\cite{vampyr})
is a Python package built on top of MRCPP\cite{mrcpp}. MRCPP is a program
package for multi-resolution analysis using MultiWavelets
(for instance, Frediani et al\cite{frediani2013fully})
developed in modern C++ with OpenMP parallelisation.
This enables fast coding in Python with even faster
running parallel C++ routines in the background. This combined with
powerful Python packages such as NumPy\cite{numpy} and SciPy\cite{SciPy}
makes it an extremely powerful tool for prototyping ideas, that can be implemented
into MRChem\cite{mrchem}. To summarize, VAMPyR can be used to prototype
new ideas. Then the VAMPyR implementation can be used as a reference when
finished product is being implemented into MRChem\cite{mrchem}. Where it can
be used to study molecular or crystal like properties. In the end reducing
the time between the idea and the publication. To make VAMPyR attractive
for chemists is a manual containing good documentation and a thorough tutorial
of how to utilize VAMPyR.

The tutorial is to be implemented in Jupyter Notebooks, which will
give an interactive and, try for yourself and see of it works,
introduciton to VAMPyR.
The tutorial should contain examples of energy calculations using
both Hartree-Fock (HF) and Density Functional Theory (DFT), both on
molecular and crystalline(@Luca usikker på ordet der) systems.

The work will conclude with a poster presentation at the ISTCP
2019\cite{istcp} in Tromso.

\section{Budget}



\bibliographystyle{unsrt}
\bibliography{ref.bib}
\end{document}
