\documentclass[12pt]{article}
\usepackage{parskip}
\usepackage[superscript, biblabel]{cite}
\usepackage[nolist]{acronym}

\begin{acronym}
  \acro{DFT}{Density Functional Theory}
  \acro{HF}{Hartree Fock}
  \acro{SCF}{self-consistent field}
  \acro{MW}{Multiwavelet}
  \acro{MRCPP}{MultiResolution Computation Program Package}
  \acro{MRChem}{MultiResolution Chemistry}
  \acro{vampyr}[VAMPyR]{Very Accurate Multiresolution Python Routines}
  \acro{KS}{Kohn-Sham}
  \acro{API}{Application Programming Interface}
\end{acronym}

\begin{document}

\title{VAMPyR}

\maketitle

\section{Background}

A recent development in quantum chemistry is the use of real-space grid
methods to achieve high precision results within the framework of
\ac{DFT}. One such method makes use of \acp{MW} as a
basis\cite{harrison2003multiresolution}
and is currently being used and further developed at the Hylleraas
center in Troms\o \, in the group of Ass.~Prof.~Luca Frediani.

The group has developed two codes the last 10 years the \ac{MRChem}\cite{mrchem} and
the \ac{MRCPP}\cite{mrcpp}.
\ac{MRChem}, is
a numerical real-space code for molecular electronic structure calculations
within the \ac{SCF} approximation of quantum chemistry. This is built on
\ac{MRCPP}, which is a general purpose numerical mathematics library
based on multiresolution analysis and the \ac{MW} basis which provide
low-scaling algorithms as well as rigorous error control in numerical
computations. These codes are written in C++, a low level programming language.
C++ developed codes are both fast and robust, but for visualization and
rapid code development it is inferior Python.

To facilitate rapid code development,
the \ac{MRChem} group has started to develop a thrid code,
\ac{vampyr}\cite{vampyr}. \ac{vampyr} gives access to the
fast and parallel \ac{API}
of \ac{MRChem} and \ac{MRCPP} from within Python. Using Pyhon developers can
combine \ac{vampyr} with libaries such as NumPy for linear algebra, matplotlib\ref{}
and avogadro\ref{} for visualization of orbitals, densities, molecular structures
and unit-cells.

\section{Project goals}

We will desing and implemnt a set of tutorials and examples on how
\ac{vampyr} can be integrated with \ac{MRChem} and \ac{MRCPP} to
facilitate code development and to provide powerful analysis tools for
the end-user, such as for example, visualising orbitals, densities, molecular structures and unit-cells.

As a starting point We will consider the following types of tutorials.

\begin{enumerate}
    \item Tutorial/template of how orbitals, densities, molecular structure and unit cells
    can be visualized using matplotlib and Avogadro.
    \item Tutorial of how to solve the \ac{HF} and \ac{KS} equations using \ac{VAMPyR}
    \item Tutorial of how to include solvation effects in \ac{VAMPyR}.
\end{enumerate}

The first one is aim at the end-user whereas the other two are more
for program developers.

\section{Implementation}

VAMPyR has extensive tests of the implementation. The student can use these
combined with the documentations of \ac{MRCPP} and \ac{MRChem} to implement
the first milestone.

The second milestone can be achieved by studying how exchange functionals
are used within \ac{MRChem}. From this examples of how to use it in VAMPyR
can be generated.

The third milestone can be achieved by studying implementations of the
\ac{HF} and \ac{KS} equations within \ac{MRChem}, alongside this the
thesis of S. R. Jensen can be used as a reference to explain the
theoretical aspects of the implementation.

The forth milestone can be used by studying the solvation implementation
in \ac{MRChem}, combind with the master thesis of Gabriel Gerez explaining
the theoretical aspects of the implementation.

At last the student will present the work on a poster at ISTCP-2019

The work will be supervised by Ass.~Prof.~Luca Frediani and Research Fellow
Magnar Bj\o rgve.

\section{Budget}



\bibliographystyle{unsrt}
\bibliography{ref.bib}
\end{document}
