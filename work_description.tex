\documentclass[12pt]{article}
\usepackage{parskip}
\usepackage[superscript, biblabel]{cite}
\usepackage[nolist]{acronym}

\begin{acronym}
  \acro{DFT}{Density Functional Theory}
  \acro{HF}{Hartree Fock}
  \acro{SCF}{self-consistent field}
  \acro{MW}{Multiwavelet}
  \acro{MRCPP}{MultiResolution Computation Program Package}
  \acro{MRChem}{MultiResolution Chemistry}
  \acro{VAMPyR}{Very Accurate Multiresolution Python Routines}
  \acro{KS}{Kohn-Sham}
\end{acronym}

\begin{document}

\title{Bla bla}

\maketitle

\section{Background}

A recent development in quantum chemistry is the use real-space grid
methods to achieve high precision results within the framework of
\ac{DFT}. One such method makes use of \ac{MW} as a
basis\cite{harrison2003multiresolution}
and is currently being used and further developed at the Hylleraas
center in Tromsø in the group of Ass.~Prof.~Luca Frediani.

The group is involved in developing three codes. The \ac{MRChem}\cite{mrchem} code, which is
a numerical real-space code for molecular electronic structure calculations
within the \ac{SCF} approximations of quantum chemistry. This is built on
the \ac{MRCPP}\cite{mrcpp} code, which is a general purpose numerical mathematics library
based on multiresolution analysis and the \ac{MW} basis which provide
low-scaling algorithms as well as rigorous error control in numerical
computations. These codes are written in C++, a low level programming language.
So users spends time on technical aspects of with the programming
language rather than focusing on the chemistry. A solution to this problem
is the \ac{VAMPyR}\cite{vampyr}. \ac{VAMPyR} is a python program packaged built upon \ac{MRCPP}
and \ac{MRChem}. \ac{VAMPyR} can be used to develop everything, that
can be developed using \ac{MRCPP} and \ac{MRChem}.Though
python the users has access to powerful libraries for computation and
visualization such as NumPy, SciPy and Pyplot.

\section{Project goals}
The goal of this project is to create \ac{VAMPyR} tutorials in Jupyter Notebooks.
These will ensure an easy and interactive entry-point
for new users. Containing five milestones:

\begin{enumerate}
    \item Tutorial of how to use  \ac{VAMPyR} with NumPy, SciPy and Pyplot.
    \item Tutorial of how to use use exchange functionals from \ac{VAMPyR}
    \item Tutorial of how to solve \ac{HF} and \ac{KS} equations using \ac{VAMPyR}
    \item Tutorial of how \ac{VAMPyR} can be used to take solvation effects into
    account.
    \item Poster presenting the work at ISTCP-2019\cite{istcp}
\end{enumerate}

\section{Implementation}

VAMPyR has extensive tests of the implementation. The student can use these
combined with the documentations of \ac{MRCPP} and \ac{MRChem} to implement
the first milestone.

The second milestone can be achieved by studying how exchange functionals
are used within \ac{MRChem}. From this examples of how to use it using VAMPyR
can be generated.

The third milestone can be achieved by studying implementations of the
\ac{HF} and \ac{KS} equations within \ac{MRChem}, alongside this the
thesis of S. R. Jensen can be used as a reference to explain the
theoretical aspects of the implementation.

The forth milestone can be used by studying the solvation implementation
in \ac{MRChem}, combind with the master thesis of Gabriel Gerez (to be submitted) explaining
the theoretical aspects of the implementation.

At last the student will present the work on a poster at ISTCP-2019.


The work will be supervised by Ass.~Prof.~Luca Frediani and Research Fellow
Magnar Bj\o rgve.

\section{Budget}

\begin{tabular}{l, c}
    ISTCP participation fee: && 2250 NOK
\end{tabular}



\bibliographystyle{unsrt}
\bibliography{ref.bib}
\end{document}
